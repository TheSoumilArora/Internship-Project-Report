\section{Conclusion}
Throughout my internship at Statcon Electronics India Ltd., I gained extensive
knowledge and practical experience in understanding, testing, and designing
LLC based resonant converters. This journey involved understanding and applying
the transition from simulation to real hardware implementation.

\noindent
LLC Resonant converters offers several advantages over conventional SMPS based
converters, such as:
\begin{itemize}
    \item \textbf{Higher Efficiency}: Resonant converters are more efficient than non-resonant converters because they can operate at a higher switching frequency. This reduces the amount of energy lost in the switching process.
    \item \textbf{Low Ripple}: Resonant converters can produce a lower ripple output voltage than non-resonant converters. This is because the resonant circuit acts as a filter, removing high-frequency components from the output voltage.
    \item \textbf{Wide Bandwidth}: Resonant converters can operate over a wide bandwidth. This means that they can be used with a variety of input and output voltages.
    \item \textbf(Zero Voltage Switching): Resonant converters can switch at zero voltage. This reduces the stress on the switching devices and increases their lifespan.
    \item \textbf{Soft Switching}: Resonant converters can switch at zero current. This reduces the stress on the switching devices and increases their lifespan.
    \item \textbf{High Power Density}: Resonant converters have a high power density. This means that they can deliver a lot of power in a small package.
    \item \textbf{High Reliability}: Resonant converters are more reliable than non-resonant converters. This is because they have fewer components and operate at a lower temperature.
    \item \textbf{Low Electromagnetic Interference (EMI)}: Resonant converters produce less electromagnetic interference than non-resonant converters. This is because they operate at a higher frequency.
    \item \textbf{High Power Factor}: Resonant converters have a high power factor. This means that they can deliver power more efficiently. (Our Design incorporates a PFC before the LLC, so the power factor is at the supply side is close to 1)
    \item \textbf{Active Voltage and Current Control}: Active voltage and current control which ensures that the output voltage and current are always in control and the converter is always in a stable state.
\end{itemize}

\noindent
Moreover, the hands-on experience with simulation tools like LtSpice
and practical circuits involving high DC voltage (upto 400V) and the complete
motherboard circuit has enriched my understanding of both simulation and
hardware aspects of power electronics. This dual exposure has equipped
me with a holistic view of the challenges and solutions in modern power
converters technology.

\section{Future Scope}
The knowledge and skills developed during this internship have broad
applications in various fields, including:

\begin{itemize}
    \item \textbf{Integrated Power Supplies}: The primary use case of this LLC based resonant converter is the Integrated Power Supplies (IPS) for the railways for our company in which 4 (or more) of such converters are wired up in parallel and then used to power the load.
    \item \textbf{Telecom Power Supplies}: These converters can be used in telecom power supplies where the load is highly dynamic and the power supply needs to be efficient and reliable.
    \item \textbf{Battery Charging}: Since our design has variable output voltage with active voltage and current control, we can use it for battery charging applications in electric vehicles.
    \item \textbf{Renewable Energy Systems}: These converters can be used in renewable energy systems like solar and wind power systems where the input voltage is highly variable and the power supply needs to be efficient.
    \item \textbf{Medical Equipment}: These converters can be used in medical equipment where the power supply needs to be reliable and efficient.
    \item \textbf{Industrial Automation}: These converters can be used in industrial automation where the power supply needs to be efficient and reliable.
\end{itemize}

\noindent
Overall, this internship has been a valuable learning experience, providing me
with the technical expertise and practical skills needed to contribute
effectively to the field of power electronics. The insights gained have not only
broadened my knowledge base but also ignited a passion for further exploration
and innovation in this dynamic and impactful field.